\documentclass[a4paper,12pt]{report} % Puedes cambiar el tamaño del papel y la fuente
\usepackage[spanish]{babel}  % Idioma español
\usepackage[dvipsnames]{xcolor}        % Para cambiar colores
\usepackage{lipsum}        % Para generar texto de ejemplo
\usepackage{pagecolor}     % Para cambiar el color de fondo
\usepackage{tikz}                 % Para crear líneas y gráficos
\usepackage{hyperref} % Crear enlaces dentro del documento

% Opciones de color para los enlaces
\hypersetup{
    colorlinks=true,
    linkcolor=black,      % Color de los enlaces
    urlcolor=blue,
    filecolor=magenta,
    citecolor=red
}

\begin{document}

% Crear la portada
\begin{titlepage}
    % Cambiar el fondo a negro y el texto a blanco solo en la portada
    \pagecolor{MidnightBlue}
    \color{white}
    \centering
    {\Huge \textbf{Algebra Lineal Básica}} \\[2.5cm] % Título principal
    {\Large Hernandez Pacheco Moises} \\[0.5cm]       % Nombre del autor
    {\Large Crystal} \\[6cm]       % Nombre del autor
    \color{SkyBlue}
    {\Large Matemáticas Aplicadas \& Computación} \\[1cm]
    \color{white}
    \textbf{\large Universidad Nacional Autónoma de México} \\[0.5cm]   % Institución o universidad
    \textbf{\large FES Acatlán} \\[1cm]   % Facultad
    % Dibujar líneas decorativas con TikZ
    \vfill
    \begin{tikzpicture}[xscale=3] % Ajustar el escalado de las líneas
        \draw[white, thick] (-2,0) -- (2,0); % Línea superior
        \draw[white, thick] (-2,-0.5) -- (2,-0.5); % Línea inferior
    \end{tikzpicture}
    
    \vfill
    {\large \today}                              % Fecha actual
\end{titlepage}

% Restablecer colores a los predeterminados para el resto del documento
\pagecolor{white}
\color{black}

\tableofcontents % Índice de contenido (opcional)

\newpage
\renewcommand{\thechapter}{\Roman{chapter}}
\part{Introducción}
\renewcommand{\thechapter}{\arabic{chapter}}
\chapter{Introducción}
\lipsum[1-3] % Generar texto de ejemplo

% =================== CONTENIDO DEL LIBRO ===================
\newpage
\renewcommand{\thechapter}{\Roman{chapter}}
\part{Conceptos Iniciales}
\renewcommand{\thechapter}{\arabic{chapter}}
\chapter{Tipos de Matices}
\lipsum[1-3] % Generar texto de ejemplo
\end{document}
