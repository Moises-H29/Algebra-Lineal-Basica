\documentclass[a4paper,12pt]{report} % Puedes cambiar el tamaño del papel y la fuente
\usepackage{graphicx} % Para usar \scalebox
\usepackage{lettrine} % Para formatear texto en mayúsculas
\usepackage{fancyhdr}  % Para un header chido 
\usepackage[a4paper, top=2.5cm, bottom=2.5cm, left=3cm, right=3cm]{geometry} % margen por defecto del documento
\usepackage[spanish]{babel}  % Idioma español
\usepackage[dvipsnames]{xcolor}        % Para cambiar colores
\usepackage{lipsum}        % Para generar texto de ejemplo
\usepackage{pagecolor}     % Para cambiar el color de fondo
\usepackage{tikz}                 % Para crear líneas y gráficos
\usepackage{hyperref} % Crear enlaces dentro del documento
\usepackage{ragged2e}       % Paquete para alineación flexible
\usepackage{microtype}
\usepackage{amsmath} % Para matrices y weas xd
\setlength{\parindent}{0pt} % quita sangrías

% ============ HEADER ===========
\setlength{\headheight}{16pt} % Aumentar según sea necesario

% ============ FOOTER CHIDO ===========
\pagestyle{fancy}
\fancyfoot[L]{FES Acatlán}
\fancyfoot[C]{\thepage}
\fancyfoot[R]{\hyperlink{toc}{IR AL ÍNDICE}}
% Añadir una línea horizontal en el footer
\renewcommand{\footrulewidth}{0.4pt} % Grosor de la línea
\renewcommand{\headrulewidth}{0.4pt}  % Grosor de la línea del encabezado



% Opciones de color para los enlaces
\hypersetup{
    colorlinks=true,
    linkcolor=black,      % Color de los enlaces
    urlcolor=blue,
    filecolor=magenta,
    citecolor=red
}

\begin{document}
\newgeometry{top=4cm, bottom=4cm, left=4cm, right=4cm}
% Crear la portada
\begin{titlepage}
    % Cambiar el fondo a negro y el texto a blanco solo en la portada
    \pagecolor{MidnightBlue}
    \color{white}
    \centering
    {\Huge \textbf{Algebra Lineal Básica}} \\[2.5cm] % Título principal
    {\Large Hernandez Pacheco Moises} \\[0.5cm]       % Nombre del autor
    {\Large Ramírez Hernandez Crystal} \\[6cm]       % Nombre del autor
    \color{SkyBlue}
    {\Large Matemáticas Aplicadas \& Computación} \\[1cm]
    \color{white}
    \textbf{\large Universidad Nacional Autónoma de México} \\[0.5cm]   % Institución o universidad
    \textbf{\large FES Acatlán} \\[1cm]   % Facultad
    % Dibujar líneas decorativas con TikZ
    \vfill
    \begin{tikzpicture}[xscale=3] % Ajustar el escalado de las líneas
        \draw[white, thick] (-2,0) -- (2,0); % Línea superior
        \draw[white, thick] (-2,-0.5) -- (2,-0.5); % Línea inferior
    \end{tikzpicture}
    
    \vfill
    {\large \today}                              % Fecha actual
\end{titlepage}

\restoregeometry % cambiar margenes a los iniciales

% Restablecer colores a los predeterminados para el resto del documento
\pagecolor{white}
\color{black}
\hypertarget{toc}{} % Target al indice
\tableofcontents % Índice de contenido (opcional)

\newpage
\renewcommand{\thechapter}{\Roman{chapter}}
\part{Introducción}
\renewcommand{\thechapter}{\arabic{chapter}}
\chapter{Introducción}
Hola Mundo

% =================== CONTENIDO DEL LIBRO ===================
\newpage
\renewcommand{\thechapter}{\Roman{chapter}}
\part{Conceptos Iniciales}
\renewcommand{\thechapter}{\arabic{chapter}}
\chapter{Tipos de Matices}

% ========== MATRIZ CUADRADA ==========
\section{Matriz Cuadrada}
Es una matriz que consta del mismo número de filas que de columnas.
En símbolos, es aquella en que \textit{m = n}. Al referirse a una matriz cuadrada de orden \textit{(n,n)}, se dice simplemente que es una \textit{matriz cuadrada de orden \textbf{n}}.\\
Por ejemplo:
\\ \\
\textit{a)}  la matriz \(\mathbf{R} = \left[ 7 \right]\), es una matriz cuadrada de orden 1;\\ \\
\textit{b)}  la matriz\\ 

\textbf{S} =
$\begin{bmatrix}
3 & 1 \\
-2 & 1
\end{bmatrix}$
es una matriz cuadrada de orden 2;\\ \\
\\\textit{c)}  la matriz\\ 

\textbf{T} =
$\begin{bmatrix}
0 & 0 & 0 \\
1 & 1 & 1 \\
2 & 2 & 2
\end{bmatrix}$
es una matriz cuadrada de orden 3; \\ \\ 
\\\textit{d)}  una matriz cuadrada de orden \textit{n}, se indica en general por\\ \\
\textbf{A} =  
$\begin{bmatrix}
a_{11} & a_{12} & \cdots & a_{1n} \\
a_{21} & a_{22} & \cdots & a_{2n} \\
\vdots & \vdots & \ddots & \vdots \\
a_{n1} & a_{n2} & \cdots & a_{nn} \\
\end{bmatrix}$
\\

% ========== MATRIZ NULA ==========
\section{Matriz Nula}
Es una matriz en que todos sus elementos son nulos. En símbolos, una matriz \(\mathbf{A} = \left[ a_{ij} \right]\) es una matriz nula, si cumple que \(a_{ij} = 0\) para todo \(i\) y \(j\).  
Se las representa con la letra \(\mathbf{O}\).
\newpage
\textit{Ejemplos:} \\ \\
\textbf{O} =
$\begin{bmatrix}
0 & 0 & 0 \\
0 & 0 & 0 \\
0 & 0 & 0
\end{bmatrix}$,\\
\vspace{0.5cm}

\textbf{O} =
$\begin{bmatrix}
0 & 0 & 0 & 0\\
0 & 0 & 0 & 0
\end{bmatrix}$,\\
\vspace{0.5cm}

\textbf{O} =
$\begin{bmatrix}
0 & 0 \\
0 & 0
\end{bmatrix}$\\

% ========== MATRIZ DIAGONAL ==========
\section{Matriz Diagonal}

Es una \textit{matriz cuadrada} en que los elementos no diagonales son todos nulos. En símbolos, una matriz \(\mathbf{A} = \left[ a_{ij} \right]_\mathit{n\times n}\) es diagonal, si se cumple que \(\mathit{a_{ij}=}\) 0, para todo \(\mathit{i\neq j}\).\\ \\
\textit{Ejemplos:}\\ \\  
\[
\begin{aligned}
\mathbf{E} &= \begin{bmatrix}
    2 & 0 & 0 \\
    0 & -3 & 0 \\
    0 & 0 & 9
\end{bmatrix}
\qquad
\mathbf{F} &= \begin{bmatrix}
    0 & 0 \\
    0 & 0
\end{bmatrix}
\qquad
\mathbf{G} &= \begin{bmatrix}
    \lambda  & 0 & 0 \\
    0 & \lambda & 0 \\
    0 & 0 & \lambda
\end{bmatrix}
\end{aligned}
\]
\\

% ========== MATRIZ ESCALAR ==========
\section{Matriz Escalar}
Es una \textit{matriz diagonal} en la que todos los elementos son iguales.\\
En símbolos, una matriz \(\mathbf{A} = \left[\mathit{a_{ij}}\right]_\mathit{n\times n}\) es una matriz escalar, si se cumple que:
\vspace{0.5cm}
\[
\mathbf{a} = \begin{cases}
    \lambda & \text{para } \mathit{i = j} \\
    0 & \text{para } \mathit{i\neq j}
    \end{cases}
\]
\\ \\
\textit{Ejemplos:} \\ 
\[
\begin{aligned}
\mathbf{J} &= \begin{bmatrix}
    -2 & 0 \\
    0 & -2
\end{bmatrix}
\qquad
\mathbf{K} &= \begin{bmatrix}
    \lambda  & 0 & 0 \\
    0 & \lambda & 0 \\
    0 & 0 & \lambda
\end{bmatrix}
\qquad
\mathbf{L} &= \begin{bmatrix}
    0 & 0 & 0 \\
    0 & 0 & 0 \\
    0 & 0 & 0
\end{bmatrix}
\end{aligned}
\]
\\ \\
No se impone ninguna condición particular sobre el valor del número \(\lambda\) puede ser un número natural, entero, racional, real, o complejo.
\\

% ========== MATRIZ IDENTIDAD ==========
\section{Matriz Identidad (o Unidad)}
Es una \textit{matriz escalar} en que todos sus elementos diagonales son iguales a la unidad.\\
Se las simboliza con \(\mathbf{I}_\mathit{n}\), en que \textit{n} indica el orden matricial, o simplemente con \textbf{I}. \\ \\
\textit{Ejemplos:} \\
\[
\begin{aligned}
\mathbf{J} &= \begin{bmatrix}
    1 & 0 \\
    0 & 1
\end{bmatrix}
\qquad
\mathbf{K} &= \begin{bmatrix}
    1  & 0 & 0 \\
    0 & 1 & 0 \\
    0 & 0 & 1
\end{bmatrix}
\qquad
\mathbf{L} &= \begin{bmatrix}
    1 & 0 & 0 & 0\\
    0 & 1 & 0 & 0\\
    0 & 0 & 1 & 0\\
    0 & 0 & 0 & 1
\end{bmatrix}
\end{aligned}
\]
\\ \\
En símbolos, se escribe
\[
\begin{aligned}
    \mathbf{I} = \left[\mathit{\mathbf{\delta}_{ij}}\right]
    \qquad
    \text{con } \mathbf{\delta}_{ij} &= \begin{cases}
        1, & \text{si } \mathit{i = j} \\
        0, & \text{si } \mathit{i\neq j}
    \end{cases}
\end{aligned}
\]
\\ \\
En razón de que la letra \textit{i} ya se ha reservado para representar las \textit{filas} de una matriz cualquiera, se acostumbra utilizar el símbolo \(\delta_{ij}\), llamada ``delta de Kronecker", para representar el elemento genérico de la unidad.\footnote{En honor al matemático alemán Leopold Kronecker (1823-1891).}\\
N\scalebox{0.8}{OTA}: Esta matriz es muy útil, como base de espacios vectoriales, en la solución de sistemas de ecuaciones lineales, y de problemas de Programación Lineal.
\\ 

% ========== MATRIZ TRIANGULAR SUPERIOR ==========
\section{Matriz Triangular Superior}
Es una matriz en que todos los elementos bajo la diagonal principal son \textbf{nulos}. En símbolos, una matriz \(\mathbf{A} = \left[\mathit{a_{ij}}\right]_\mathit{n\times n}\) es triangular superior, si se cumple que \(\mathit{a_{ij}}=0\) para todo \(\mathit{i>j}\).
\\ \\
\textit{Ejemplos:} \\ 

\[
\begin{aligned}
\mathbf{A} &= \begin{bmatrix}
    3  & 0 & -1 \\
    0 & 1 & 0 \\
    0 & 0 & \text{0.1}
\end{bmatrix},
\qquad
\mathbf{B} &= \begin{bmatrix}
    3 & -1 & 2 & 0\\
    0 & -1 & 0 & -5\\
    0 & 0 & 0 & 6\\
    0 & 0 & 0 & 2
\end{bmatrix}
\end{aligned}
\]\\ \\
Note que no se impone ninguna condición sobre los elementos situados en la diagonal principal o por encima de ella; algunos de los elementos \(\mathit{a_{ij}}\), para los que \(\mathit{i\leq j}\) pueden ser también nulos. Ese es el caso de los elementos \(\mathit{b_{14}\text{, }b_{23}\text{ y }b_{33}}\) de la matriz \textbf{B}.
\\

% ========== MATRIZ TRIANGULAR SUPERIOR ==========
\section{Matriz Triangular Inferior}
Es una matriz en que todos los elementos sobre la diagonal principal son \textbf{nulos}. En símbolos, una matriz \(\mathbf{A} = \left[\mathit{a_{ij}}\right]_\mathit{n\times n}\) es triangular inferior, si se cumple que \(\mathit{a_{ij}}=0\) para todo \(\mathit{i<j}\).
\\ \\
\textit{Ejemplos:} \\ 

\[
\begin{aligned}
\mathbf{C} &= \begin{bmatrix}
    -3 & 0 & 0 \\
    5 & 1 & 0 \\
    2 & -2 & 0
\end{bmatrix},
\qquad
\mathbf{D} &= \begin{bmatrix}
    -3 & 0 & 0 & 0\\
    5 & -1 & 0 & 0\\
    2 & -4 & 0 & 0\\
    -1 & 0 & 7 & 8
\end{bmatrix}
\end{aligned}
\]\\ \\
Note que algunos de los elementos \(\mathit{a_{ij}}\), para los que \(\mathit{i\geq j}\) pueden ser nulos; no se estipula ninguna condición especial sobre ellos. Ese es el caso de los elementos \(\mathit{d_{33}\text{ y }d_{42}}\) de la matriz \textbf{D}.
\\
% ========== MATRIZ SIMETRICA ==========
\section{Matriz Simétrica}
Es una matriz \textit{cuadrada} \(\mathbf{A} = \left[\mathit{a_{ij}}\right]_\mathit{n\times n}\) en que \(\mathit{a_{ij}} = \mathit{a_{ji}}\) para todo \textit{i}, \textit{j}.
\\ \\
\textit{Ejemplos:} \\ \\
1) Sea la matriz \(\mathbf{M} = \left[\mathit{m_{ij}}\right]_\mathit{2\times 2}\) siguiente:
\[
\mathbf{M} = \begin{bmatrix}
    -1 & 2\\
    2 & 1
\end{bmatrix}
\]\\ \\
Esta matriz es simétrica, ya que se cumple la igualdad \(\mathit{m_{12}} = \mathit{m_{21}} = 2\). \\ \\
2) Sea la matriz \(\mathbf{N} = \left[\mathit{n_{ij}}\right]_\mathit{3\times 3}\) siguiente:
\[
\mathbf{N} = \begin{bmatrix}
    1 & 2 & 1\\
    2 & 3 & -3\\
    1 & -3 & 4
\end{bmatrix}
\]\\ \\
Esta matriz es simétrica, ya que se cumplen las igualdades
\[
\begin{matrix}
\mathit{n_{12}} &= \mathit{n_{21}} &= \text{ 2}, \\
\mathit{n_{13}} & = \mathit{n_{31}} &= \text{ 1}, \\
\mathit{n_{23}} & = \mathit{n_{32}} &= \text{ -3}
\end{matrix}
\]
\\
% ========== MATRIZ ANTISIMETRICA ==========
\section{Matriz Antisimétrica}
Es una matriz \textit{cuadrada} \(\mathbf{A} = \left[\mathit{a_{ij}}\right]_\mathit{n\times n}\) en que \(\mathit{a_{ij}} = \mathit{-a_{ji}}\) para todo \textit{i} y \textit{j}.
\\ \\
\textit{Ejemplos:} \\ \\
1) Sea la matriz \(\mathbf{Q} = \left[\mathit{q_{ij}}\right]\) de orden dos:
\[
\mathbf{Q} = \begin{bmatrix}
    0 & -1\\
    1 & 0
\end{bmatrix}
\]\\ \\
Esta matriz es antisimétrica, ya que se cumple la igualdad 
\[\mathit{q_{21}} = \mathit{-q_{12}}.\] \\ \\
2) Sea la matriz \(\mathbf{R} = \left[\mathit{r_{ij}}\right]\) de orden tres:
\[
\mathbf{N} = \begin{bmatrix}
    0 & 2 & 1\\
    -2 & 0 & -3\\
    -1 & 3 & 0
\end{bmatrix}
\]\\ \\
Esta matriz es antisimétrica, ya que se cumplen las igualdades
\[
\begin{matrix}
\mathit{r_{12}} &= \mathit{-r_{21}},\\
\mathit{r_{13}} & = \mathit{-r_{31}},\\
\mathit{r_{23}} & = \mathit{-r_{32}}
\end{matrix}
\]
\\
Además, la definición de matriz antisimétrica implica que los elementos de su diagonal principal sean nulos.

\end{document}
