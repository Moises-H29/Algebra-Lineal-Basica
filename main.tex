\documentclass[a4paper,12pt]{report} % Puedes cambiar el tamaño del papel y la fuente
\usepackage{fancyhdr}  % Para un header chido 
\usepackage[margin=4cm]{geometry}
\usepackage[spanish]{babel}  % Idioma español
\usepackage[dvipsnames]{xcolor}        % Para cambiar colores
\usepackage{lipsum}        % Para generar texto de ejemplo
\usepackage{pagecolor}     % Para cambiar el color de fondo
\usepackage{tikz}                 % Para crear líneas y gráficos
\usepackage{hyperref} % Crear enlaces dentro del documento
\usepackage{ragged2e}       % Paquete para alineación flexible
\usepackage{microtype}
\usepackage{amsmath} % Para matrices y weas xd
\setlength{\parindent}{0pt} % quita sangrías

% ============ FOOTER CHIDO ===========
\pagestyle{fancy}
\fancyfoot[L]{FES Acatlán}
\fancyfoot[C]{\thepage}
\fancyfoot[R]{\hyperlink{toc}{IR AL ÍNDICE}}
% Añadir una línea horizontal en el footer
\renewcommand{\footrulewidth}{0.4pt}  % Grosor de la línea
\renewcommand{\headrulewidth}{0.4pt}  % Grosor de la línea del encabezado



% Opciones de color para los enlaces
\hypersetup{
    colorlinks=true,
    linkcolor=black,      % Color de los enlaces
    urlcolor=blue,
    filecolor=magenta,
    citecolor=red
}

\begin{document}

% Crear la portada
\begin{titlepage}
    % Cambiar el fondo a negro y el texto a blanco solo en la portada
    \pagecolor{MidnightBlue}
    \color{white}
    \centering
    {\Huge \textbf{Algebra Lineal Básica}} \\[2.5cm] % Título principal
    {\Large Hernandez Pacheco Moises} \\[0.5cm]       % Nombre del autor
    {\Large Ramírez Hernandez Crystal} \\[6cm]       % Nombre del autor
    \color{SkyBlue}
    {\Large Matemáticas Aplicadas \& Computación} \\[1cm]
    \color{white}
    \textbf{\large Universidad Nacional Autónoma de México} \\[0.5cm]   % Institución o universidad
    \textbf{\large FES Acatlán} \\[1cm]   % Facultad
    % Dibujar líneas decorativas con TikZ
    \vfill
    \begin{tikzpicture}[xscale=3] % Ajustar el escalado de las líneas
        \draw[white, thick] (-2,0) -- (2,0); % Línea superior
        \draw[white, thick] (-2,-0.5) -- (2,-0.5); % Línea inferior
    \end{tikzpicture}
    
    \vfill
    {\large \today}                              % Fecha actual
\end{titlepage}

% Restablecer colores a los predeterminados para el resto del documento
\pagecolor{white}
\color{black}
\newgeometry{top=2.5cm, bottom=2.5cm, left=2.5cm, right=2.5cm}
\hypertarget{toc}{} % Target al indice
\tableofcontents % Índice de contenido (opcional)

\newpage
\renewcommand{\thechapter}{\Roman{chapter}}
\part{Introducción}
\renewcommand{\thechapter}{\arabic{chapter}}
\chapter{Introducción}
Hola Mundo

% =================== CONTENIDO DEL LIBRO ===================
\newpage
\renewcommand{\thechapter}{\Roman{chapter}}
\part{Conceptos Iniciales}
\renewcommand{\thechapter}{\arabic{chapter}}
\chapter{Tipos de Matices}

\section{Matriz Cuadrada}
Es una matriz que consta del mismo número de filas que de columnas.
En símbolos, es aquella en que \textit{m = n}. Al referirse a una matriz cuadrada de orden \textit{(n,n)}, se dice simplemente que es una \textit{matriz cuadrada de orden \textbf{n}}.\\
Por ejemplo:
\\ \\
\textit{a)}  la matriz \(\mathbf{R} = \left[ 7 \right]\), es una matriz cuadrada de orden 1;\\ \\
\textit{b)}  la matriz\\ 

\textbf{S} =
$\begin{bmatrix}
3 & 1 \\
-2 & 1
\end{bmatrix}$
es una matriz cuadrada de orden 2;\\ \\
\\\textit{c)}  la matriz\\ 

\textbf{T} =
$\begin{bmatrix}
0 & 0 & 0 \\
1 & 1 & 1 \\
2 & 2 & 2
\end{bmatrix}$
es una matriz cuadrada de orden 3; \\ \\ 
\\\textit{d)}  una matriz cuadrada de orden \textit{n}, se indica en general por\\ \\
\textbf{A} =  
$\begin{bmatrix}
a_{11} & a_{12} & \cdots & a_{1n} \\
a_{21} & a_{22} & \cdots & a_{2n} \\
\vdots & \vdots & \ddots & \vdots \\
a_{n1} & a_{n2} & \cdots & a_{nn} \\
\end{bmatrix}$\\

\section{Matriz Nula}
Es una matriz en que todos sus elementos son nulos. En símbolos, una matriz \(\mathbf{A} = \left[ a_{ij} \right]\) es una matriz nula, si cumple que \(a_{ij} = 0\) para todo \(i\) y \(j\).  
Se las representa con la letra \(\mathbf{O}\).
\newpage
\textit{Ejemplos:} \\ \\
\textbf{O} =
$\begin{bmatrix}
0 & 0 & 0 \\
0 & 0 & 0 \\
0 & 0 & 0
\end{bmatrix}$,\\
\vspace{0.5cm}

\textbf{O} =
$\begin{bmatrix}
0 & 0 & 0 & 0\\
0 & 0 & 0 & 0
\end{bmatrix}$,\\
\vspace{0.5cm}

\textbf{O} =
$\begin{bmatrix}
0 & 0 \\
0 & 0
\end{bmatrix}$\\

\section{Matriz Diagonal}

Es una matriz cuadrada en que los elementos no diagonales son todos nulos. En símbolos, una matriz \(\mathbf{A} = \left[ a_{ij} \right]_{n\times n}\) es diagonal, si se cumple que \(\mathit{a_{ij}=}\) 0, para todo \(\mathit{i\neq j}\).\\ \\
\textit{Ejemplos:}\\ \\  
\[
\begin{aligned}
\mathbf{E} &= \begin{bmatrix}
    2 & 0 & 0 \\
    0 & -3 & 0 \\
    0 & 0 & 9
\end{bmatrix}
\qquad
\mathbf{F} &= \begin{bmatrix}
    0 & 0 \\
    0 & 0
\end{bmatrix}
\qquad
\mathbf{G} &= \begin{bmatrix}
    \lambda  & 0 & 0 \\
    0 & \lambda & 0 \\
    0 & 0 & \lambda
\end{bmatrix}
\end{aligned}
\]



\end{document}
